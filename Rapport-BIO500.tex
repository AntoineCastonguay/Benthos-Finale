% see http://info.semprag.org/basics for a full description of this template
\documentclass[]{glossa}

% possible options:
% [times] for Times font (default if no option is chosen)
% [cm] for Computer Modern font
% [lucida] for Lucida font (not freely available)
% [brill] open type font, freely downloadable for non-commercial use from http://www.brill.com/about/brill-fonts; requires xetex
% [charis] for CharisSIL font, freely downloadable from http://software.sil.org/charis/
% for the Brill an CharisSIL fonts, you have to use the XeLatex typesetting engine (not pdfLatex)
% for headings, tables, captions, etc., Fira Sans is used: https://www.fontsquirrel.com/fonts/fira-sans
% [biblatex] for using biblatex (the default is natbib, do not load the natbib package in this file, it is loaded automatically via the document class glossa.cls)
% [linguex] loads the linguex example package
% !! a note on the use of linguex: in glossed examples, the third line of the example (the translation) needs to be prefixed with \glt. This is to allow a first line with the name of the language and the source of the example. See example (2) in the text for an illustration.
% !! a note on the use of bibtex: for PhD dissertations to typeset correctly in the references list, the Address field needs to contain the city (for US cities in the format "Santa Cruz, CA")

%\addbibresource{sample.bib}
% the above line is for use with biblatex
% replace this by the name of your bib-file (extension .bib is required)
% comment out if you use natbib/bibtex

\let\B\relax %to resolve a conflict in the definition of these commands between xyling and xunicode (the latter called by fontspec, called by charis)
\let\T\relax
\usepackage{xyling} %for trees; the use of xyling with the CharisSIL font produces poor results in the branches. This problem does not arise with the packages qtree or forest.
\usepackage[linguistics]{forest} %for nice trees!
\usepackage{longtable}


\title[A subtitle goes on another line]{Rapport de l'indice de pollution
en lien avec la faune benthique.}
% Optional short title inside square brackets, for the running headers.

% \author[Paul \& Vanden Wyngaerd]% short form of the author names for the running header. If no short author is given, no authors print in the headers.
% {%as many authors as you like, each separated by \AND.
%   \spauthor{Waltraud Paul\\
%   \institute{CNRS, CRLAO}\\
%   \small{105, Bd. Raspail, 75005 Paris\\
%   waltraud.paul@ehess.fr}
%   }
%   \AND
%   \spauthor{Guido Vanden Wyngaerd \\
%   \institute{KU Leuven}\\
%   \small{Warmoesberg 26, 1000 Brussel\\
%   guido.vandenwyngaerd@kuleuven.be}
%   }%
% }

\author[Paul \& Vanden Wyngaerd]{
    \spauthor{Claudiane Bondu\\
  \institute{}\\
  \small{}
  }%
  \AND  \spauthor{Antoine Castonguay\\
  \institute{}\\
  \small{}
  }%
  \AND  \spauthor{Juliette Robin\\
  \institute{}\\
  \small{}
  }%
  }

\usepackage{natbib}


% tightlist command for lists without linebreak
\providecommand{\tightlist}{%
  \setlength{\itemsep}{0pt}\setlength{\parskip}{0pt}}






\begin{document}


\sffamily
\maketitle



\rmfamily

%  Body of the article
\hypertarget{introduction}{%
\section{INTRODUCTION}\label{introduction}}

L'étude porte sur le lien entre l'indice de population, sa richesse et
son abondance de la faune benthique des rivières du Québec. Cette
dernière a été réalisée par des étudiants au baccalauréat en écologie à
l'Université de Sherbrooke. Elle participe à contribuer de façon
significative à la compréhension de l'écosystème aquatique de cette
région. Les rivières du Québec abritent une diversité de vie benthique,
qui joue un grand rôle dans le maintien de l'équilibre écologique et de
la santé des écosystèmes aquatiques. Cependant, certaines de ces espèces
sont sensibles au niveau de pollution de l'eau ce qui fait d'eux de bons
indicateurs de la qualité de l'eau. Dans cette étude, les étudiants se
sont penchés sur l'abondance de chaque site ainsi que de leur richesse
en lien avec des facteurs environnent. En examinant cette relation, ils
visent à récolter des informations qui permettront d'aider à
sensibiliser les riverains de la région sur l'effet de la pollution sur
la biodiversité aquatique.

\hypertarget{muxe9thodologie}{%
\section{MÉTHODOLOGIE}\label{muxe9thodologie}}

Les sites d'échantillonnage mesurent 100 m et devront se situer à au
moins 100 m en amont ou en aval (si les conditions en amont sont
inadéquates) d'une zone urbanisée comme route ou d'un pont. Les coups de
filet doivent être fait à des vitesses du courant différents, à des
profondeurs différentes, quelques-uns en bordure et d'autres plus au
centre pour favoriser la capture d'une plus grande diversité de taxons.
Au total, 20 coups de filet seront réalisés. L'échantillonnage commence
en aval du site et s'effectue à l'aide d'un filet troubleau. Il doit
être bien rincé avant de passer à la station suivante pour éviter la
contamination. À la station, une zone d'échantillonner d'une longueur de
50 cm sur une largeur de 30 cm en amont du filet est délimité. Ce
dernier sera enfoncé légèrement dans le substrat afin de ne laisser
aucun organisme en dessous et son ouverture fait face au courant.
Pendant 30 secondes, une personne va frotter les roches et les débris
avec les mains à l'intérieur de la surface à échantillonner. Lorsque les
endroits sont trop profonds, les pieds peuvent être utilisés. Retirer le
filet à contre-courant pour éviter que des organismes s'échappent avec
le courant. Le contenu de chaque coup de filet est transféré dans un
seau à fond grillagé à chaque coup de filet. Le seau est laissé dans une
eau peu profonde et calme pour d'éviter l'assèchement de l'échantillon.
À la fin des 20 coups de filet, les gros débris sont jeté après avoir
été inspectés afin de trouver des organismes et les remettre dans
l'échantillon. Ensuite, l'échantillon doit être rincé à l'eau claire
afin d'enlever les sédiments fins en enfonçant le seau dans l'eau et on
remue délicatement l'échantillon pour le débarrasser des particules les
plus fines, lesquelles s'échapperont par le fond grillagé. L'opération
peut être répétée plusieurs fois. Ensuite, laisser l'eau s'égoutter de
l'échantillon puis transférer le dans un contenant auquel est ajouté de
l'alcool à 95 \%. L'eau restant contenue dans l'échantillon diluera la
solution jusqu'au niveau approprié, soit 70 à 80 \% d'alcool.
L'échantillon total constitué du benthos et des détritus devrait avoir
un volume d'environ 1 litre. Les contenants doivent être étiquetés. Une
étiquette doit être collé sur le pot et doit indiquer la date, le nom de
la rivière, le numéro de la région hydrographique (annexe 1), le numéro
de la station, le nombre de contenants et le nom du bassin versant
principal. Une seconde étiquette est mise à l'intérieur des contenants
en papier imperméable. Il faut s'assurer que les contenants sont bien
scellés et que les organismes sont entièrement recouverts par l'alcool.
Par la suite, l'identification des macroinvertébrés benthiques des
échantillons sont effectués en laboratoire.

\hypertarget{ruxe9sultats}{%
\section{RÉSULTATS}\label{ruxe9sultats}}

\hypertarget{discussion}{%
\section{DISCUSSION}\label{discussion}}

\renewcommand\refname{BIBLIOGRAPHIE}
\bibliography{sample.bib}


\end{document}
